\documentclass[10pt,a4paper]{article}
\usepackage[utf8]{inputenc}
\usepackage[brazilian]{babel}
\usepackage[T1]{fontenc}
\usepackage{geometry}
\usepackage{enumitem}
\usepackage{titlesec}
\usepackage{fancyhdr}
\usepackage{xcolor}
\usepackage{multicol}
\usepackage{ulem}

% ===== CONFIGURAÇÃO DE PÁGINA =====
\geometry{top=2cm, bottom=2.5cm, left=1.5cm, right=1.5cm}

% ===== RODAPÉ =====
\pagestyle{fancy}
\fancyhf{}
\fancyfoot[L]{\small AGENTE SOCIOEDUCATIVO - MASCULINO - \thepage}
\fancyfoot[R]{\small www.ioconcursos.com.br}
\renewcommand{\headrulewidth}{0pt}
\renewcommand{\footrulewidth}{0pt}

% ===== FORMATAÇÃO =====
\titleformat{\section}{\Large\bfseries}{\thesection}{1em}{}
\titleformat{\subsection}{\large\bfseries}{\thesubsection}{1em}{}

\setlength{\parindent}{0pt}
\setlength{\parskip}{0.5em}
\setlength{\columnsep}{1cm}

% ===== COMANDOS PERSONALIZADOS =====

% Cabeçalho de disciplina
\newcommand{\disciplina}[1]{%
  \vspace{1em}
  {\large\textbf{\uline{#1}}}
  \vspace{0.5em}
}

% Gabarito correto (opcional - comente para não mostrar)
% \newcommand{\correto}[1]{{\color{red}\textbf{(Correta: #1)}}}

% Questão anulada (informativo)
\newcommand{\anulada}{{\color{red}\textbf{(Questão anulada)}}}

% ===== INÍCIO DO DOCUMENTO =====
\begin{document}

% ===== PÁGINA DE TÍTULO (OPCIONAL) =====
\begin{titlepage}
    \centering
    \vspace*{2cm}
    {\Huge\bfseries Simulado IASES 2025\par}
    \vspace{1cm}
    {\Large Agente Socioeducativo\par}
    \vspace{1.5cm}
    {\large Questões 01-70\par}
    \vspace{2cm}
    {\large Nome do Simulado\par}
    \vfill
    {\large Espírito Santo - Brasil\par}
    {\large \today\par}
\end{titlepage}

\tableofcontents
\newpage

% ===== INÍCIO DO LAYOUT EM DUAS COLUNAS =====
\begin{multicols}{2}

\section*{Concurso Público IASES 2022}
\textsc{AGENTE SOCIOEDUCATIVO - MASCULINO}

\rule{\columnwidth}{0.5pt}

% ============================================
% LÍNGUA PORTUGUESA (10 questões)
% ============================================

\disciplina{Língua Portuguesa}

\small

% ----- TEXTO BASE (se houver) -----
\textbf{Texto Base - Questões 1 a 5}

\textbf{Título do Texto}

\textit{Informações sobre origem, autor, data...}

Texto completo aqui. Lorem ipsum dolor sit amet, consectetur adipiscing elit...

\textit{\small Disponível em: https://fonte.com.br Acesso em: dd/mm/aaaa}

% ----- QUESTÕES -----

\textbf{Questão 01}

Enunciado da questão número um...

(A) Primeira alternativa.

(B) Segunda alternativa.

(C) Terceira alternativa.

(D) Quarta alternativa.

(E) Quinta alternativa.

\textbf{Questão 02}

Enunciado da questão número dois...

(A) Primeira alternativa.

(B) Segunda alternativa.

(C) Terceira alternativa.

(D) Quarta alternativa.

(E) Quinta alternativa.

\textbf{Questão 03}

Enunciado da questão número três...

(A) Primeira alternativa.

(B) Segunda alternativa.

(C) Terceira alternativa.

(D) Quarta alternativa.

(E) Quinta alternativa.

% ... Continue até questão 10

% ============================================
% RACIOCÍNIO LÓGICO (5 questões)
% ============================================

\disciplina{Raciocínio Lógico}

\textbf{Questão 11}

Enunciado...

(A) Alternativa A.

(B) Alternativa B.

(C) Alternativa C.

(D) Alternativa D.

(E) Alternativa E.

% ... Continue até questão 15

% ============================================
% INFORMÁTICA (5 questões)
% ============================================

\columnbreak  % Força quebra de coluna

\disciplina{Informática}

\textbf{Questão 16}

Enunciado...

(A) Alternativa A.

(B) Alternativa B.

(C) Alternativa C.

(D) Alternativa D.

(E) Alternativa E.

% ... Continue até questão 20

% ============================================
% CONHECIMENTOS ESPECÍFICOS (50 questões)
% ============================================

\disciplina{Conhecimentos Específicos}

\textbf{Questão 21}

Enunciado...

(A) Alternativa A.

(B) Alternativa B.

(C) Alternativa C.

(D) Alternativa D.

(E) Alternativa E.

% ... Continue até questão 70

% ===== EXEMPLO DE QUESTÃO COM V/F =====
%
% \textbf{Questão XX}
%
% Analise as assertivas com V(Verdadeiro) ou F(Falso).
%
% (\_\_) Primeira afirmativa.
%
% (\_\_) Segunda afirmativa.
%
% (\_\_) Terceira afirmativa.
%
% Marque a alternativa com a sequência CORRETA:
%
% (A) V, V, F.
%
% (B) F, V, V.
%
% (C) V, F, V.
%
% (D) F, F, F.
%
% (E) V, V, V.

% ===== EXEMPLO DE QUESTÃO ANULADA =====
%
% \textbf{Questão XX}
%
% \anulada
%
% Enunciado da questão...

% ===== FIM DO LAYOUT EM DUAS COLUNAS =====
\end{multicols}

\end{document}
