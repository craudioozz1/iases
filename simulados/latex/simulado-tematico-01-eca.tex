\documentclass[10pt,a4paper]{article}
\usepackage[utf8]{inputenc}
\usepackage[brazilian]{babel}
\usepackage[T1]{fontenc}
\usepackage{geometry}
\usepackage{enumitem}
\usepackage{titlesec}
\usepackage{fancyhdr}
\usepackage{xcolor}
\usepackage{multicol}
\usepackage{ulem}

% Configuração de página
\geometry{top=2cm, bottom=2.5cm, left=1.5cm, right=1.5cm}

% Rodapé personalizado
\pagestyle{fancy}
\fancyhf{}
\fancyfoot[L]{\small AGENTE SOCIOEDUCATIVO - MASCULINO - \thepage}
\fancyfoot[R]{\small www.ioconcursos.com.br}
\renewcommand{\headrulewidth}{0pt}
\renewcommand{\footrulewidth}{0pt}

% Configuração de colunas
\setlength{\parindent}{0pt}
\setlength{\parskip}{0.5em}
\setlength{\columnsep}{1cm}

% Comandos personalizados
\newcommand{\disciplina}[1]{%
  \vspace{1em}
  {\large\textbf{\uline{#1}}}
  \vspace{0.5em}
}

\begin{document}

% Cabeçalho
\begin{center}
{\Large\textbf{Simulado Temático 01 - ECA}}\\
\vspace{0.3cm}
{\large Estatuto da Criança e do Adolescente}\\
\vspace{0.2cm}
\textsc{CONCURSO PÚBLICO IASES 2025}\\
\textsc{AGENTE SOCIOEDUCATIVO - MASCULINO}\\
\end{center}

\rule{\textwidth}{0.5pt}

\textbf{INSTRUÇÕES:}
\begin{itemize}[nosep]
\item Leia atentamente cada questão
\item Marque apenas UMA alternativa por questão
\item Use o cartão de respostas ao final
\item NÃO consulte material durante o simulado
\item Após terminar, confira o gabarito comentado
\end{itemize}

\rule{\textwidth}{0.5pt}

% INÍCIO DO LAYOUT EM DUAS COLUNAS
\begin{multicols}{2}

\disciplina{Conhecimentos Específicos - ECA}

\small

\textbf{Questão 01}

De acordo com o Estatuto da Criança e do Adolescente, considera-se criança a pessoa até:

(A) Dez anos de idade incompletos.

(B) Doze anos de idade incompletos.

(C) Catorze anos de idade incompletos.

(D) Dezesseis anos de idade incompletos.

\textbf{Questão 02}

Segundo o ECA, considera-se adolescente a pessoa com idade entre:

(A) Dez e dezesseis anos.

(B) Doze e dezoito anos.

(C) Catorze e dezoito anos.

(D) Dezesseis e vinte e um anos.

\textbf{Questão 03}

De acordo com o artigo 4º do ECA, é dever de quem assegurar, com absoluta prioridade, a efetivação dos direitos referentes à vida, à saúde, à alimentação, à educação, ao esporte, ao lazer, à profissionalização, à cultura, à dignidade, ao respeito, à liberdade e à convivência familiar e comunitária da criança e do adolescente?

(A) Apenas da família e do poder público.

(B) Apenas do poder público e da sociedade em geral.

(C) Da família, da comunidade, da sociedade em geral e do poder público.

(D) Exclusivamente do poder público, através de suas instituições.

\textbf{Questão 04}

Os casos de suspeita ou confirmação de castigo físico, de tratamento cruel ou degradante e de maus-tratos contra criança ou adolescente serão obrigatoriamente comunicados:

(A) À Delegacia de Polícia da respectiva localidade.

(B) Ao Conselho Tutelar da respectiva localidade.

(C) Ao Ministério Público Estadual.

(D) À Vara da Infância e Juventude.

\textbf{Questão 05}

Segundo o artigo 13 do ECA, a comunicação dos casos de suspeita ou confirmação de maus-tratos ao Conselho Tutelar ocorre:

(A) Com prejuízo de outras providências legais, que ficam suspensas até decisão do Conselho.

(B) Sem prejuízo de outras providências legais.

(C) Apenas quando esgotadas todas as outras possibilidades de solução.

(D) Somente após confirmação médica dos maus-tratos.

\textbf{Questão 06}

De acordo com o artigo 18-A do ECA, a criança e o adolescente têm o direito de ser educados e cuidados sem o uso de:

(A) Apenas castigo físico, sendo permitido o tratamento cruel para fins educativos.

(B) Castigo físico ou de tratamento cruel ou degradante.

(C) Qualquer forma de disciplina ou correção.

(D) Castigo físico, salvo quando aplicado pelos pais ou responsáveis.

\textbf{Questão 07}

Para os efeitos do ECA, considera-se castigo físico a ação de natureza disciplinar ou punitiva aplicada com o uso da força física sobre a criança ou o adolescente que resulte em:

(A) Apenas lesão corporal.

(B) Sofrimento físico ou lesão.

(C) Somente sofrimento psicológico.

(D) Exclusivamente lesão grave ou gravíssima.

\textbf{Questão 08}

Segundo o ECA, considera-se tratamento cruel ou degradante a conduta que, em relação à criança ou ao adolescente:

(A) Humilhe, ameace gravemente ou ridicularize.

(B) Apenas provoque lesão física.

(C) Somente resulte em morte.

(D) Cause exclusivamente dano material.

\textbf{Questão 09}

De acordo com o artigo 22 do ECA, aos pais incumbe o dever de:

(A) Sustento, guarda e educação dos filhos menores.

(B) Apenas sustento e educação dos filhos menores.

(C) Apenas guarda dos filhos menores.

(D) Sustento, guarda, educação e lazer dos filhos menores.

\textbf{Questão 10}

Os dirigentes de estabelecimentos de ensino fundamental comunicarão ao Conselho Tutelar os casos de:

(A) Qualquer falta dos alunos, justificada ou não.

(B) Apenas maus-tratos envolvendo seus alunos.

(C) Maus-tratos, reiteração de faltas injustificadas e evasão escolar, e elevados níveis de repetência.

(D) Exclusivamente situações de violência física comprovada.

\textbf{Questão 11}

Segundo o artigo 56 do ECA, a comunicação ao Conselho Tutelar sobre reiteração de faltas injustificadas e evasão escolar deve ocorrer:

(A) Imediatamente após a primeira falta.

(B) Após esgotados os recursos escolares.

(C) Apenas quando houver mais de 50\% de faltas.

(D) Somente mediante autorização dos pais.

\textbf{Questão 12}

De acordo com o artigo 67 do ECA, ao adolescente empregado, aprendiz, em regime familiar de trabalho, aluno de escola técnica, assistido em entidade governamental ou não governamental, é vedado o trabalho:

(A) Noturno, perigoso ou insalubre.

(B) Apenas noturno.

(C) Apenas perigoso.

(D) Em qualquer circunstância.

\textbf{Questão 13}

As férias do adolescente empregado ou aprendiz devem ter duração não inferior a:

(A) Quinze dias.

(B) Vinte dias.

(C) Trinta dias.

(D) Quarenta e cinco dias.

\textbf{Questão 14}

Segundo o ECA, as férias do adolescente trabalhador devem coincidir, sempre que possível, com:

(A) As férias dos pais ou responsáveis.

(B) As férias escolares.

(C) O período de recesso da empresa.

(D) O início do ano civil.

\textbf{Questão 15}

Considera-se ato infracional:

(A) Apenas a conduta descrita como crime.

(B) A conduta descrita como crime ou contravenção penal.

(C) Apenas comportamentos violentos.

(D) Exclusivamente crimes hediondos praticados por adolescentes.

\textbf{Questão 16}

De acordo com o artigo 104 do ECA, são penalmente inimputáveis os menores de:

(A) Dezesseis anos.

(B) Dezoito anos.

(C) Vinte e um anos.

(D) Catorze anos.

\textbf{Questão 17}

Para os efeitos do ECA, ao determinar a inimputabilidade penal, deve ser considerada a idade do adolescente:

(A) À data do julgamento.

(B) À data da sentença.

(C) À data do fato.

(D) À data da denúncia pelo Ministério Público.

\textbf{Questão 18}

Verificada a prática de ato infracional, a autoridade competente NÃO poderá aplicar ao adolescente a seguinte medida:

(A) Advertência.

(B) Prestação de serviços à comunidade.

(C) Prisão em regime fechado.

(D) Internação em estabelecimento educacional.

\textbf{Questão 19}

São medidas socioeducativas previstas no artigo 112 do ECA, EXCETO:

(A) Obrigação de reparar o dano.

(B) Liberdade assistida.

(C) Detenção por tempo determinado.

(D) Semiliberdade.

\textbf{Questão 20}

De acordo com o ECA, quantas são as medidas socioeducativas que podem ser aplicadas ao adolescente que pratica ato infracional?

(A) Cinco medidas.

(B) Seis medidas.

(C) Sete medidas.

(D) Oito medidas.

\textbf{Questão 21}

A internação constitui medida privativa da liberdade, sujeita aos princípios de:

(A) Exemplaridade, dissuasão e ressocialização.

(B) Brevidade, excepcionalidade e respeito à condição peculiar de pessoa em desenvolvimento.

(C) Punição, reeducação e reinserção social.

(D) Segregação, disciplina e obediência.

\textbf{Questão 22}

Segundo o ECA, em relação às atividades externas na medida de internação, é correto afirmar que:

(A) São proibidas em qualquer hipótese.

(B) São obrigatórias.

(C) Serão permitidas a critério da equipe técnica da entidade, salvo expressa determinação judicial em contrário.

(D) Dependem exclusivamente de autorização judicial.

\textbf{Questão 23}

De acordo com o artigo 122 do ECA, a medida de internação só poderá ser aplicada quando:

(A) Tratar-se de qualquer ato infracional cometido pelo adolescente.

(B) Tratar-se de ato infracional cometido mediante grave ameaça ou violência a pessoa, por reiteração no cometimento de outras infrações graves, ou por descumprimento reiterado e injustificável da medida anteriormente imposta.

(C) O adolescente for maior de 16 anos.

(D) Houver pedido expresso dos pais ou responsáveis.

\textbf{Questão 24}

A entidade que desenvolve programa de internação deve ter as seguintes características, EXCETO:

(A) Em entidade exclusiva para adolescentes.

(B) Em local próximo ou mais próximo possível ao domicílio dos pais ou responsável.

(C) Em grandes complexos para melhor aproveitamento de recursos.

(D) Pequenas unidades e grupos reduzidos.

\textbf{Questão 25}

São direitos do adolescente privado de liberdade, entre outros, os seguintes, EXCETO:

(A) Entrevistar-se pessoalmente com o representante do Ministério Público.

(B) Peticionar diretamente a qualquer autoridade.

(C) Recusar visitas de familiares sem justificativa.

(D) Receber visitas ao menos semanalmente.

\textbf{Questão 26}

De acordo com o artigo 124 do ECA, o adolescente privado de liberdade tem direito a permanecer internado:

(A) Apenas na capital do Estado.

(B) Na mesma localidade ou na mais próxima ao domicílio de seus pais ou responsável.

(C) Em qualquer localidade, a critério do juiz.

(D) Exclusivamente em estabelecimentos federais.

\textbf{Questão 27}

Segundo o ECA, é dever do Estado zelar pela integridade física e mental dos internos, cabendo-lhe adotar:

(A) Medidas de punição exemplar.

(B) Medidas adequadas de contenção e segurança.

(C) Isolamento total dos internos.

(D) Restrição de todos os direitos durante a internação.

\textbf{Questão 28}

O Conselho Tutelar é:

(A) Órgão permanente e autônomo, não jurisdicional.

(B) Órgão jurisdicional vinculado ao Poder Judiciário.

(C) Órgão temporário e subordinado ao Ministério Público.

(D) Órgão jurisdicional vinculado à Defensoria Pública.

\textbf{Questão 29}

Entre as atribuições do Conselho Tutelar previstas no artigo 136 do ECA, NÃO se inclui:

(A) Atender crianças e adolescentes nas hipóteses previstas em lei, aplicando medidas de proteção.

(B) Atender e aconselhar os pais ou responsável, aplicando as medidas previstas em lei.

(C) Julgar e condenar adolescentes que praticarem atos infracionais.

(D) Encaminhar ao Ministério Público notícia de fato que constitua infração administrativa ou penal contra os direitos da criança ou adolescente.

\textbf{Questão 30}

Deixar o médico, professor ou responsável por estabelecimento de atenção à saúde e de ensino fundamental, pré-escola ou creche, de comunicar à autoridade competente os casos de que tenha conhecimento, envolvendo suspeita ou confirmação de maus-tratos contra criança ou adolescente, sujeita o infrator à pena de:

(A) Detenção de 6 meses a 2 anos.

(B) Multa de 3 a 20 salários de referência, aplicando-se o dobro em caso de reincidência.

(C) Advertência, seguida de multa em caso de reincidência.

(D) Suspensão do exercício profissional por 30 dias.

\end{multicols}

\newpage

% Cartão de Respostas
\begin{center}
{\Large\textbf{CARTÃO DE RESPOSTAS}}
\end{center}

\textbf{Nome:} \underline{\hspace{10cm}}

\textbf{Data:} \underline{\hspace{3cm}} \textbf{Horário de início:} \underline{\hspace{2cm}} \textbf{Horário de término:} \underline{\hspace{2cm}}

\vspace{1cm}

\begin{multicols}{2}
\small
01. [ A ] [ B ] [ C ] [ D ]

02. [ A ] [ B ] [ C ] [ D ]

03. [ A ] [ B ] [ C ] [ D ]

04. [ A ] [ B ] [ C ] [ D ]

05. [ A ] [ B ] [ C ] [ D ]

06. [ A ] [ B ] [ C ] [ D ]

07. [ A ] [ B ] [ C ] [ D ]

08. [ A ] [ B ] [ C ] [ D ]

09. [ A ] [ B ] [ C ] [ D ]

10. [ A ] [ B ] [ C ] [ D ]

11. [ A ] [ B ] [ C ] [ D ]

12. [ A ] [ B ] [ C ] [ D ]

13. [ A ] [ B ] [ C ] [ D ]

14. [ A ] [ B ] [ C ] [ D ]

15. [ A ] [ B ] [ C ] [ D ]

\columnbreak

16. [ A ] [ B ] [ C ] [ D ]

17. [ A ] [ B ] [ C ] [ D ]

18. [ A ] [ B ] [ C ] [ D ]

19. [ A ] [ B ] [ C ] [ D ]

20. [ A ] [ B ] [ C ] [ D ]

21. [ A ] [ B ] [ C ] [ D ]

22. [ A ] [ B ] [ C ] [ D ]

23. [ A ] [ B ] [ C ] [ D ]

24. [ A ] [ B ] [ C ] [ D ]

25. [ A ] [ B ] [ C ] [ D ]

26. [ A ] [ B ] [ C ] [ D ]

27. [ A ] [ B ] [ C ] [ D ]

28. [ A ] [ B ] [ C ] [ D ]

29. [ A ] [ B ] [ C ] [ D ]

30. [ A ] [ B ] [ C ] [ D ]

\end{multicols}

\vspace{1cm}

\textbf{Contagem de acertos:} \underline{\hspace{2cm}}/30

\textbf{Pontuação:} \underline{\hspace{2cm}}/60 pontos

\textbf{Percentual:} \underline{\hspace{2cm}}\%

\textbf{Aprovado:} [ ] SIM \hspace{1cm} [ ] NÃO (mínimo 50\%)

\vspace{1cm}

\begin{center}
\rule{0.8\textwidth}{0.5pt}

\textbf{GABARITO OFICIAL}

\rule{0.8\textwidth}{0.5pt}
\end{center}

\begin{verbatim}
01-B  02-B  03-C  04-B  05-B  06-B  07-B  08-A  09-A  10-C
11-B  12-A  13-C  14-B  15-B  16-B  17-C  18-C  19-C  20-C
21-B  22-C  23-B  24-C  25-C  26-B  27-B  28-A  29-C  30-B
\end{verbatim}

\end{document}
