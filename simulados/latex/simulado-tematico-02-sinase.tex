\documentclass[10pt,a4paper]{article}
\usepackage[utf8]{inputenc}
\usepackage[brazilian]{babel}
\usepackage[T1]{fontenc}
\usepackage{geometry}
\usepackage{enumitem}
\usepackage{titlesec}
\usepackage{fancyhdr}
\usepackage{xcolor}
\usepackage{multicol}
\usepackage{ulem}

% Configuração de página
\geometry{top=2cm, bottom=2.5cm, left=1.5cm, right=1.5cm}

% Rodapé personalizado
\pagestyle{fancy}
\fancyhf{}
\fancyfoot[L]{\small AGENTE SOCIOEDUCATIVO - MASCULINO - \thepage}
\fancyfoot[R]{\small www.ioconcursos.com.br}
\renewcommand{\headrulewidth}{0pt}
\renewcommand{\footrulewidth}{0pt}

% Configuração de colunas
\setlength{\parindent}{0pt}
\setlength{\parskip}{0.5em}
\setlength{\columnsep}{1cm}

% Comandos personalizados
\newcommand{\disciplina}[1]{%
  \vspace{1em}
  {\large\textbf{\uline{#1}}}
  \vspace{0.5em}
}

\begin{document}

% Cabeçalho
\begin{center}
{\Large\textbf{Simulado Temático 02 - SINASE}}\\
\vspace{0.3cm}
{\large Sistema Nacional de Atendimento Socioeducativo}\\
\vspace{0.2cm}
\textsc{CONCURSO PÚBLICO IASES 2025}\\
\textsc{AGENTE SOCIOEDUCATIVO - MASCULINO}\\
\end{center}

\rule{\textwidth}{0.5pt}

\textbf{INSTRUÇÕES:}
\begin{itemize}[nosep]
\item Leia atentamente cada questão
\item Marque apenas UMA alternativa por questão
\item Use o cartão de respostas ao final
\item NÃO consulte material durante o simulado
\item Após terminar, confira o gabarito comentado
\end{itemize}

\rule{\textwidth}{0.5pt}

% INÍCIO DO LAYOUT EM DUAS COLUNAS
\begin{multicols}{2}

\disciplina{Conhecimentos Específicos - SINASE}

\small

\textbf{Questão 01}

De acordo com o artigo 1º da Lei 12.594/2012, esta Lei institui o:

(A) Sistema Nacional de Justiça da Infância e Juventude.

(B) Sistema Nacional de Atendimento Socioeducativo (Sinase).

(C) Sistema Brasileiro de Proteção Integral à Criança.

(D) Sistema Nacional de Medidas Socioeducativas e Protetivas.

\textbf{Questão 02}

A Lei do SINASE regulamenta a execução das medidas socioeducativas destinadas a:

(A) Crianças e adolescentes em situação de risco.

(B) Adolescentes que pratiquem ato infracional.

(C) Crianças que pratiquem ato infracional.

(D) Adultos jovens entre 18 e 21 anos.

\textbf{Questão 03}

De acordo com o SINASE, compete aos Estados:

(A) Criar e desenvolver programas para execução de todas as medidas socioeducativas.

(B) Criar, desenvolver e manter programas para a execução das medidas socioeducativas de semiliberdade e internação.

(C) Apenas fiscalizar os programas de atendimento socioeducativo.

(D) Criar programas exclusivamente para medidas em meio aberto.

\textbf{Questão 04}

Segundo o SINASE, compete aos Municípios:

(A) Criar e manter programas de internação.

(B) Criar, desenvolver e manter programas de atendimento para a execução das medidas socioeducativas de semiliberdade e internação.

(C) Criar, desenvolver e manter programas de atendimento para a execução das medidas socioeducativas em meio aberto.

(D) Apenas cofinanciar programas estaduais de atendimento.

\textbf{Questão 05}

As medidas socioeducativas em meio aberto, de competência municipal, são:

(A) Advertência e obrigação de reparar o dano.

(B) Liberdade assistida e prestação de serviços à comunidade.

(C) Semiliberdade e liberdade assistida.

(D) Prestação de serviços à comunidade e semiliberdade.

\textbf{Questão 06}

De acordo com o artigo 35 do SINASE, a execução das medidas socioeducativas reger-se-á por quantos princípios?

(A) Sete princípios.

(B) Oito princípios.

(C) Nove princípios.

(D) Dez princípios.

\textbf{Questão 07}

Entre os princípios que regem a execução das medidas socioeducativas, previsto no artigo 35 do SINASE, NÃO se inclui:

(A) Legalidade.

(B) Excepcionalidade da intervenção judicial.

(C) Exemplaridade e dissuasão.

(D) Proporcionalidade em relação à ofensa cometida.

\textbf{Questão 08}

De acordo com o artigo 35, inciso I, do SINASE, o princípio da legalidade estabelece que:

(A) O adolescente pode receber qualquer tratamento, conforme a gravidade do ato.

(B) O adolescente não poderá receber tratamento mais gravoso do que o conferido ao adulto.

(C) O adolescente sempre receberá tratamento menos gravoso que o adulto.

(D) O tratamento do adolescente será sempre exemplar para a sociedade.

\textbf{Questão 09}

Segundo o artigo 35, inciso III, do SINASE, deve-se dar prioridade a práticas ou medidas que sejam:

(A) Punitivas e atendam ao interesse da sociedade.

(B) Restaurativas e, sempre que possível, atendam às necessidades das vítimas.

(C) Exemplares para outros adolescentes.

(D) Retributivas em relação ao dano causado.

\textbf{Questão 10}

Entre os princípios da execução das medidas socioeducativas previsto no SINASE, inclui-se:

(A) Segregação do adolescente infrator.

(B) Fortalecimento dos vínculos familiares e comunitários no processo socioeducativo.

(C) Isolamento social como medida de proteção.

(D) Afastamento compulsório da família em todos os casos.

\textbf{Questão 11}

De acordo com o SINASE, as medidas de proteção, advertência e reparação do dano, quando aplicadas de forma isolada, serão executadas:

(A) Em autos apartados do processo de conhecimento.

(B) Nos próprios autos do processo de conhecimento.

(C) Apenas após elaboração do Plano Individual de Atendimento.

(D) Exclusivamente por equipe multidisciplinar.

\textbf{Questão 12}

As medidas de prestação de serviços à comunidade, liberdade assistida, semiliberdade e internação serão executadas:

(A) Nos próprios autos do processo de conhecimento.

(B) Sem necessidade de elaboração de plano de atendimento.

(C) Por meio de Plano Individual de Atendimento (PIA).

(D) Apenas mediante decisão do Conselho Tutelar.

\textbf{Questão 13}

As medidas socioeducativas de liberdade assistida, semiliberdade e internação deverão ser reavaliadas no máximo a cada:

(A) Três meses.

(B) Seis meses.

(C) Doze meses.

(D) Dezoito meses.

\textbf{Questão 14}

Segundo o artigo 42 do SINASE, a autoridade judiciária poderá, se necessário, designar audiência no prazo máximo de:

(A) Cinco dias.

(B) Dez dias.

(C) Quinze dias.

(D) Vinte dias.

\textbf{Questão 15}

De acordo com o artigo 42 do SINASE, na reavaliação das medidas socioeducativas, a autoridade judiciária deverá cientificar, EXCETO:

(A) O defensor.

(B) O Ministério Público.

(C) A vítima do ato infracional.

(D) O adolescente e seus pais ou responsável.

\textbf{Questão 16}

A extinção da medida socioeducativa ocorrerá quando o adolescente completar:

(A) Dezoito anos.

(B) Dezenove anos.

(C) Vinte anos.

(D) Vinte e um anos.

\textbf{Questão 17}

Segundo o artigo 46, § 1º, do SINASE, a medida socioeducativa aplicada ao adolescente poderá ser extinta por:

(A) Pedido expresso dos pais ou responsáveis.

(B) Aplicação de pena privativa de liberdade, provisória ou definitivamente.

(C) Decisão administrativa do programa de atendimento.

(D) Solicitação unilateral do adolescente.

\textbf{Questão 18}

São direitos do adolescente submetido à execução de medida socioeducativa, EXCETO:

(A) Ser acompanhado por seus pais ou responsável e por seu defensor na audiência.

(B) Ser informado dos atos processuais.

(C) Recusar atendimento técnico sem justificativa.

(D) Peticionar diretamente a qualquer autoridade.

\textbf{Questão 19}

De acordo com o artigo 49, inciso V, do SINASE, o adolescente tem direito de:

(A) Impedir a participação da família no processo socioeducativo.

(B) Solicitar a substituição de seu defensor.

(C) Solicitar revisão da medida a qualquer tempo.

(D) Escolher livremente o programa de atendimento.

\textbf{Questão 20}

Constituem direitos do adolescente em medida socioeducativa:

(A) Apenas segurança e habitação.

(B) Educação de qualidade e profissionalização.

(C) Exclusivamente alimentação e saúde.

(D) Somente atividades religiosas obrigatórias.

\textbf{Questão 21}

Segundo o artigo 50 do SINASE, a educação do adolescente em medida socioeducativa deve ser ministrada, preferencialmente:

(A) Em sistema exclusivo para infratores.

(B) No sistema regular de ensino.

(C) Apenas em regime de educação a distância.

(D) Em sistema especial segregado.

\textbf{Questão 22}

De acordo com o artigo 51 do SINASE, as entidades de atendimento socioeducativo devem reavaliar periodicamente cada caso, com intervalo máximo de:

(A) Três meses.

(B) Quatro meses.

(C) Seis meses.

(D) Doze meses.

\textbf{Questão 23}

O Plano Individual de Atendimento (PIA) deverá ser elaborado no prazo máximo de:

(A) Quinze dias a contar da data do ingresso do adolescente no programa de atendimento.

(B) Trinta dias a contar da data do ingresso do adolescente no programa de atendimento.

(C) Quarenta e cinco dias a contar da data do ingresso do adolescente no programa de atendimento.

(D) Sessenta dias a contar da data do ingresso do adolescente no programa de atendimento.

\textbf{Questão 24}

A elaboração do Plano Individual de Atendimento (PIA) compete à equipe técnica do respectivo programa de atendimento, com a participação efetiva:

(A) Apenas da direção do programa.

(B) Exclusivamente do Ministério Público.

(C) Do adolescente e de sua família.

(D) Somente do juiz da Vara da Infância.

\textbf{Questão 25}

O Plano Individual de Atendimento (PIA) é obrigatório para as seguintes medidas socioeducativas:

(A) Apenas advertência e reparação do dano.

(B) Prestação de serviços à comunidade, liberdade assistida, semiliberdade e internação.

(C) Somente internação.

(D) Todas as medidas previstas no ECA, sem exceção.

\textbf{Questão 26}

Segundo o SINASE, incumbe ao Ministério Público, entre outras atribuições:

(A) Executar diretamente as medidas socioeducativas.

(B) Fiscalizar os programas de atendimento socioeducativo.

(C) Administrar as unidades de internação.

(D) Aplicar as medidas socioeducativas aos adolescentes.

\textbf{Questão 27}

De acordo com o SINASE, incumbe à Defensoria Pública:

(A) Julgar os atos infracionais.

(B) Prestar orientação jurídica e defesa dos adolescentes.

(C) Executar as medidas socioeducativas em meio aberto.

(D) Fiscalizar as entidades de atendimento.

\textbf{Questão 28}

Segundo o artigo 35, inciso VI, do SINASE, o princípio da individualização considera:

(A) Apenas a gravidade do ato infracional.

(B) Exclusivamente a idade do adolescente.

(C) A idade, capacidades e circunstâncias pessoais do adolescente.

(D) Somente os antecedentes infracionais.

\textbf{Questão 29}

O princípio da mínima intervenção, previsto no artigo 35, VII, do SINASE, determina que a medida socioeducativa:

(A) Deve ser a mais gravosa possível para gerar exemplaridade.

(B) Deve restringir-se ao necessário para a realização dos objetivos da medida.

(C) Deve sempre ser aplicada em regime fechado.

(D) Pode ser aplicada além do necessário para punir adequadamente.

\textbf{Questão 30}

De acordo com o artigo 35, inciso VIII, do SINASE, constitui princípio da execução das medidas socioeducativas a não discriminação do adolescente, notadamente em razão de:

(A) Apenas etnia e classe social.

(B) Somente orientação religiosa e política.

(C) Etnia, gênero, nacionalidade, classe social, orientação religiosa, política ou sexual, ou associação ou pertencimento a qualquer minoria ou status.

(D) Exclusivamente idade e situação econômica.

\end{multicols}

\newpage

% Cartão de Respostas
\begin{center}
{\Large\textbf{CARTÃO DE RESPOSTAS}}
\end{center}

\textbf{Nome:} \underline{\hspace{10cm}}

\textbf{Data:} \underline{\hspace{3cm}} \textbf{Horário de início:} \underline{\hspace{2cm}} \textbf{Horário de término:} \underline{\hspace{2cm}}

\vspace{1cm}

\begin{multicols}{2}
\small
01. [ A ] [ B ] [ C ] [ D ]

02. [ A ] [ B ] [ C ] [ D ]

03. [ A ] [ B ] [ C ] [ D ]

04. [ A ] [ B ] [ C ] [ D ]

05. [ A ] [ B ] [ C ] [ D ]

06. [ A ] [ B ] [ C ] [ D ]

07. [ A ] [ B ] [ C ] [ D ]

08. [ A ] [ B ] [ C ] [ D ]

09. [ A ] [ B ] [ C ] [ D ]

10. [ A ] [ B ] [ C ] [ D ]

11. [ A ] [ B ] [ C ] [ D ]

12. [ A ] [ B ] [ C ] [ D ]

13. [ A ] [ B ] [ C ] [ D ]

14. [ A ] [ B ] [ C ] [ D ]

15. [ A ] [ B ] [ C ] [ D ]

\columnbreak

16. [ A ] [ B ] [ C ] [ D ]

17. [ A ] [ B ] [ C ] [ D ]

18. [ A ] [ B ] [ C ] [ D ]

19. [ A ] [ B ] [ C ] [ D ]

20. [ A ] [ B ] [ C ] [ D ]

21. [ A ] [ B ] [ C ] [ D ]

22. [ A ] [ B ] [ C ] [ D ]

23. [ A ] [ B ] [ C ] [ D ]

24. [ A ] [ B ] [ C ] [ D ]

25. [ A ] [ B ] [ C ] [ D ]

26. [ A ] [ B ] [ C ] [ D ]

27. [ A ] [ B ] [ C ] [ D ]

28. [ A ] [ B ] [ C ] [ D ]

29. [ A ] [ B ] [ C ] [ D ]

30. [ A ] [ B ] [ C ] [ D ]

\end{multicols}

\vspace{1cm}

\textbf{Contagem de acertos:} \underline{\hspace{2cm}}/30

\textbf{Pontuação:} \underline{\hspace{2cm}}/60 pontos

\textbf{Percentual:} \underline{\hspace{2cm}}\%

\textbf{Aprovado:} [ ] SIM \hspace{1cm} [ ] NÃO (mínimo 50\%)

\vspace{1cm}

\begin{center}
\rule{0.8\textwidth}{0.5pt}

\textbf{GABARITO OFICIAL}

\rule{0.8\textwidth}{0.5pt}
\end{center}

\begin{verbatim}
01-B  02-B  03-B  04-C  05-B  06-C  07-C  08-B  09-B  10-B
11-B  12-C  13-B  14-B  15-C  16-D  17-B  18-C  19-C  20-B
21-B  22-C  23-C  24-C  25-B  26-B  27-B  28-C  29-B  30-C
\end{verbatim}

\end{document}
